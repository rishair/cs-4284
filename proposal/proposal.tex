\documentclass{article}

\usepackage[margin=0.8in]{geometry}

\pagestyle{empty}

\begin{document}

\begin{center}
{\huge Systems and Networking Capstone Project}

{\large Analyzing scalability of IRC command and control network}
\end{center} 

\section*{Members}
\begin{itemize}
\item Kevin Malachowski (kchowski@vt.edu)
\item Alex Shipley (alexs91@vt.edu)
\item Reese Moore (ram@vt.edu)
\item Rishi Ishairzay (windowasher@gmail.com)
\end{itemize}

\section*{Problem Statement}

Rackspace needs to be able to reliably control a large number of geographically
distributed compute nodes.

\section*{Potential Solutions}
\begin{itemize}
\item Use RabbitMQ\footnote{\texttt http://www.rabbitmq.com/} to send messages
between a ``master'' server and its ``slaves,'' delegating reliability of commands
being executed to the master.
\item IRC command and control network, where every node subscribes to specific
channels and executes commands sent to those channels.
\end{itemize}

\noindent After talking to Paul Voccio of Rackspace, he suggested that we use the straight
IRC solution. Gabe Westmass agreed for now, but mentioned that we may need to
move to a hybrid RabbitMQ-IRC solution to help with scalability.

\section*{Objectives}
\begin{enumerate}
\item Reliability of delivery and execution (correctness)
\item Scalability
\item Speed
\end{enumerate}

\section*{Timeline}
\begin{itemize}
\item Write Bot monitoring software for VMs
\item Set up Rackspace machines
    \begin{itemize}
    \item Install simple IRC servers on two of the test machines (so we can test
    linking them together in a network)
    \item Install our Bot software on the remaining nodes
    \end{itemize}
\item Develop method for scalability testing
    \begin{itemize}
    \item Possible solutions:
        \begin{itemize}
        \item Run multiple bots on each VM
        \end{itemize}
    \end{itemize}
\item Report on our findings regarding the scalability of our solution.
\end{itemize}

\end{document}
